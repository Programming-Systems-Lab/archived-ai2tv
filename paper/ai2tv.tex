% to compile this file into a pdf or ps use the following
% 1) latex whitepaper.tex
% for ps: 2) dvips -K whitepaper -o
% for pdf: 2) ps2pdf whitepaper.ps whitepaper.pdf
\documentclass[12pt]{article}
\usepackage{fancyheadings}
\usepackage{doublespace}
\usepackage{enumerate}
\usepackage{graphics}

% \usepackage{doublespace}

% commands used to control margins
\usepackage{geometry}
\geometry{verbose,letterpaper,lmargin=45mm,rmargin=45mm}
% \textheight=9.3in
\topmargin=.2in

% command used to control between-line spacing
\renewcommand{\baselinestretch}{1}

\pagestyle{fancy}
\cfoot{}
\lhead{}
\rhead{\rm\thepage\ of \pageref{'LastPage'}}
\chead{\emph{Evaluating QoS of Distributed Video Synchrony}}

\title{Using Workflow to Optimize QoS for Collaborative Client Video
Synchronization}

\author{
 \textbf{Dan B. Phung, Peppo Valetto, Gail Kaiser, Suhit Gupta}\\
Computer Science Department\\
Columbia University\\
\{phung, valetto, kaiser, suhit\}@cs.columbia.edu
}

\date{
\parbox[b][0ex]{0em}{\hspace*{-12.5em}\raisebox{37ex}{\fbox{For
submission to \emph{ACM-MM 2004}, due 12:00 AM EDT: April 05, 2004.}}}}

\begin{document}

\begin{singlespace}
\maketitle
\end{singlespace}

% PAPER SYNOPSIS
% motivation: 
% - online courses more popular
% - limited support for distributed collaboration
% - network resource disparity
% -> thus, we give you ai2tv
%
% background:
% - show trend in online courses and current state on collaborative support
% - video, or any type of distributed collaborative synchronized work
% - discuss briefly history of two ai2tv designs and imps
%   1) describe motivations for sys 1 architectural design
%   2) describe flaws of sys 1
%   -> segue into ai2tv system 2
% - describe kender's semantic compression
% - summarize achievements
% 
% evaluation methods
% - explain scoring system
% - 
% 
% results
% - give statistical results of sys 1
% - showed ai2tv w/ and w/o workflow each compares: 
%                : one graph for baseline comparison
%                : one graph for penalties
% 1) synchronization was successful
% 2) using workflow, QoS could be significantly increased
% 
% conclusion
% - our system can be used as an aid to support collaborative
% video viewing for students groups.  
% 

% (does this need a name? SYCOVI (SYnchronized COllaborative Video?)
\begin{abstract}\noindent
The increasing popularity of distance learning and online courses has
highlights the lack of support for collaborative tools available for
student groups.  Participants in these groups usually have a disparity
in network resources which can hamper productive dialog.  We design
and implement a system that allows geographically dispersed
participants to collaboratively view a video synchronously.  This
system includes an autonomic workflow controller that responds to
client feedback and adjusts the quality of the video according to the
resources of each client.  Using this system we show that we can
successfully synchronize video for distributed clients.  We also show
that by using an autonomic workflow controller momentary increases in
bandwidth can be exploited to raise the quality of video while
transitory decreases in bandwidth can be buffered against to retain
video integrity.
\end{abstract}

\textbf{keywords:} synchronized collaborative video, autonomic
workflow,

\section{Introduction} \label{sec:intro}

A major shift in educational paradigms is the use of online courses,
which have become increasingly more popular \cite{beller,doe}.  Due to
the ease in scheduling alongside a work schedule, one of the common
type of students of online courses are the non-traditional working
student that is trying to further her or his career through higher
certifications \cite{burgess:trends}.  This student population usually
varies geographically because online courses allow students from any
region to enroll.  The dispersed nature of the class prevents students
from physically holding study sessions, which is a common and
productive practice.  Support for collaboration is a major concern in
courses where group work is encouraged \cite{wells:collab} yet there
are few tools that support distributed collaboration
\cite{burgess:trends}.

The area of collaboration that our contributions apply to is
collaborative video viewing.  In this scenario, students that are
dispersed across different geographical regions can view lecture
videos simultaneously.  One proposed use of this technology is to
facilitate group study sessions to review lecture material.  While
within a video viewing session, participants can play, pause, stop and
goto certain frames synchronously as a group.

Using video on the internet requires relatively high bandiwdth
resources.  Network traffic is usually erratic, which can lead to lost
video content.  The lost content can either be good or bad, depending
on the redundancy of video material.  To eliminate this uncertainty,
we use a semantic compression package \cite{tiecheng} developed in
another lab to extract the semantically rich material.  Using a
selected time window, the package summarizes the content over that
time and extracts semantically rich keyframes.  We discuss this work
more extensively in section \ref{background}.

Though all our videos are semantically rich, the playback may seem
unnatural because only keyframes are shown.  Whether the video plays
more natural depends on the number of keyframes extracted, which is a
function of the time window length.  In order to provide the best
quality video to clients, we adjust the semantic compression algoritm
to create more or less keyframes depending on the current
bandwidth\footnote{for performance reasons, we actually precompute the
keyframes for several bandwidth levels}.

In this paper we present two designs and implementations.  The first
approach, which we will call system 1, used a peer to peer
synchronization scheme that is common in some gaming platforms
\cite{findme}.  Due to some flaws in this system, it was unable to
synchronize the video consistently, so we redesigned and implemented
another system (system 2) that took into account the mistakes made in
system 1.  System 1 uses a common event bus and an autonomic workflow
controller to respond to feedback from each of the clients and fine
tune them when possible.

We will compare the two systems and discuss the key features of the
system that successfully synchronized the video for multiple clients.
We also show that using the workflow component allows clients to view
a better quality video.

\subsection{Background} 

doing the p2p events thing?
Also, why the keeping of time through sleep methods?  another
main detractor from the system is that it was run within another
processor and memory intensive system.  


Background and context

AI2TV stands for "Adaptive Internet Interactive Team Video". AI2TV is
a project that aims at developing a collaborative virtual environment
for distributed team work, which includes an infrastructure for the
provision over geographic networks of multimedia content relevant to
the work carried out by the team, such as audio/video recordings of
group discussions and decisions, informational and educational events,
etc. The application domain of election is that of long-distance
education, for remotizing attendance and review of class lectures,
group study, and the various phases of collaborative team assignments,
such as software development projects. Notice how AI2TV differs from
many existing infrastructures for providing educational content in a
networked context because of its explicit focus on supporting the team
work aspects of on-line education.

A number of speculative, design and technological challenges underlie
the main AI2TV ideas. The issues that are most relevant to the work
presented in this document lie mainly in the area of
Computer-Supported Collaborative Work (CSCW), and how dynamic software
adaptation can aid computerized CSCW tools to meet their goal of
efficiently and effectively supporting collaborative work practices.

One of the most compelling requirements for the content provision
system of AI2TV is the support for the synchronized watching of some
streamed video clip by all members of a geographically distributed
team across their different equipment and networking capabilities,
including support for video operations like fast forward, rewind,
seek, etc: imagine a scenario in which the team decides to review
together a portion of a past lecture, in order to solve some
difficulty in their project assignment or to clarify some notion.

Given that team members can be dispersed over the Internet, and may
enjoy very diverse connectivity, ranging for example from 28.8k modem,
to DSL, to cable, to T1 lines, the multimedia content they must use
for their review session is to be delivered over heterogeneous
Internet links to heterogeneous platforms. Moreover, in such a setup
the communication and computing resources available to each user may
widely and quickly vary in the course the team work session. In
contrast with those potential difficulties, the collaboration can be
effective only in case the fruition of the streamed video clip remains
well synchronized, so that users, who are enabled by their clients to
discuss the material among each other as they see it, have a natural
group experience (like they were co-located in class, or watching a
tape together sitting in a study room) and do not incur in
misunderstandings, waste of time, or other serious inconveniences
during their review session.

AI2TV addresses this problem on the one hand by providing multimedia
content in multiple versions, with different levels of semantic
compression (citation?), and on the other hand by using the process /
workflow technology made available by Workflakes to dynamically adapt
its content provision infrastructure. Dynamic adaptation in this case
is directed at modifying a combination of server and client
configurations, data caching strategies and video management schemes,
to accommodate and harmonize varying latencies, throughputs, client
processing power, and server work loads. All of that - as will become
more evident as the implementation of the case study is described
below - must be completed by Workflakes within narrow time boundaries
(in the order of seconds or less), given the soft-real time nature of
the application and the kind of adaptation that must be effected.

That first application of process-based dynamic adaptation in the
AI2TV system is termed the short-term or client synchronization
workflow, to distinguish it from the other possible applications,
which are introduced below.

Another context in AI2TV, in which process / workflow technology plays
a significant part is the organization of the work of the team as well
as its individual members, in accord with an agenda containing a
schedule of group events (e.g., virtual study "meetings") and work
deadlines. In AI2TV, a workflow is used to model and guide the
activities of the distributed team along that planned schedule. The
typology of that workflow is in general that of a classic
human-oriented process, whose stakeholders are persons and whose goal
is to facilitate and guide the collaboration among those persons;
furthermore, the workflow is likely to span its activities across a
relatively long term, i.e. in terms of weeks, days, or hours in the
most demanding cases.

Given those characteristics that kind of workflow should not be
seemingly concerned with any dynamic software adaptation
issues. However, there are peculiarities intrinsic to the presence and
use of multimedia content in the workflow that demand for the
introduction of dynamic adaptation aspects, which become intertwined
with the coordination of team activities.

Multimedia content must be treated by the workflow as both a new type
of artifact and an additional kind of resource, albeit an expensive
one that must be carefully organized and managed. One thing that this
workflow must do is to plan and orchestrate the distribution of copies
of such multimedia artifacts to each individual team member in a
timely fashion. Ideally, all artifacts would be entirely transferred
to all clients in advance, before a planned event begins. That ideal
situation would avoid the need for streaming any information on the
fly, and would thus largely circumvent the problems that require the
intervention of the synchronization workflow, at least for
planned-ahead joint work events (for virtual meetings that are set up
and initiated with no or little advance notice, the client
synchronization workflow remains completely relevant).

In the real world, the typical situation is likely to be somewhere in
between the two extremes above. Given that, and also in the view of
the high variability of the factors that may influence or hinder the
ability to make available in advance the necessary artifacts, such as
connectivity, servers, memory storage space in the clients, CPU load
on clients, and more, there is a need for AI2TV to adapt on the fly
even in the long-term context, in order to overcome connectivity and
capability idiosyncrasies and maximize the amount of data prefetched
and immediately available to clients at the beginning of a group
event. That can be resolved with a combination of content pre-fetching
and caching techniques that are orchestrated via a pre-fetching
workflow, which kicks in as part of the long-term scheduling
workflow. The pre-fetching workflow must also graciously turn control
over to the synchronization workflow whenever a group event begins,
for keeping in check and adapting the synchronized delivery and
presentation of the material across clients.

Besides the long- and the short- term, there is also a medium-term
option for dynamically adapting the provision of multimedia content in
AI2TV. That can occur whenever, during the synchronized fruition of
some multimedia stream, the group decides to pause or interrupt the
viewing. That opens a window of opportunity for loading additional
material in clients' caches. The logic of this opportunistic
pre-fetching workflow is akin to that of the long-term pre-fetching
workflow, but it must operate within a time frame that is closer to
that of the client synchronization workflow.


\subsection{Design} \label{sec:contrib}

\subsection{Implementation} \label{sec:contrib}

\subsection{Experimental Setup} \label{sec:eval}

We only focus on network bandwidth as the commodity resource.

To evaluate our implementations, we devised a scoring scheme
to determine the amount of synchronization.  

- define baseline: (bandwidth, framerate)

- explain scoring metric.  

- explain missed frames

\section{Results} \label{sec:results}

We show that system 1 was unsuccesful in synchronizing
the clients.  This failure stemmed from implementation 
mistakes such as relying on the sleep method as a quantum
for keeping time.  In addition, the overhead of the
peer-to-peer messaging scheme weighed down the system.  The
possible sources of error in this scheme is depending on
propagation delays to be consistent across all clients.


System 2 was able to overcome these failures by depending on
NTP to synchronize clocks.  The sytem clock was then
used as the base to which the cilents synchronized.  The 
centralized event bus also reduced network traffic, which 
may have also contributed to the failure of system 1.

\section{Related Work} \label{sec:related}

\section{Future Work} \label{sec:future}

\section{Conclusion} \label{sec:con}

\section{Acknowledgements} \label{sec:ack}

\label{'LastPage'}
\end{document}
